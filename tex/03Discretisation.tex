%!TEX root = ../main.tex

\section{Discretisation}
\label{sec:Discretisation}

We use standard Lagrangian finite element spaces on $\Gamma$ as basis functions both for the geometry and the unknowns $\phi$ and $\frac{\partial\phi}{\partial n}=:\gamma$. The approximation can be divided in 6 main steps.

\subsection{Computational Mesh Generation}
\label{sub:computational_mesh_generation}

We define a quadrilateral computational mesh meant as a regular\footnote{For every cell $K$ there exist a mapping from a reference cell $\hat{K}$ to $K$ with positive determinant of the Jacobian.} decomposition $\Gamma_h$ of the boundary $\Gamma$. Then we progressively refine this very coarse mesh, by exploiting a proper \emph{a posteriori} error estimator, to obtain a solution with desired accuracy.

\subsection{Definition of the Discrete Spaces}
\label{sub:definition_of_the_discrete_spaces}

The right spaces to analyse Eq.~\eqref{eq:solid-angle} are
\begin{equation*}
\phi\in V=H^{1/2}(\Gamma)\qquad \gamma\in Q=H^{-1/2}(\Gamma)
\end{equation*}

In a conforming setting we choose $V_h\subseteq V,\ Q_h\subseteq Q$ as
\begin{subequations}
\label{eq:FEspaces}
\begin{align}
\label{eq:Vhspace}
V_h&:=\left\{ \phi_h\in L^2(\Gamma_h)\,:\,\phi_h\big|_K\in \mathbb{Q}^r(K),\ K\in\Gamma_h \right\}, \\
\label{eq:Qhspace} 
Q_h&:=\left\{ \gamma_h\in L^2(\Gamma_h)\,:\,\gamma_h\big|_K\in \mathbb{Q}^s(K),\ K\in\Gamma_h \right\}.
\end{align}
\end{subequations}


It can be shown that using the same FE discretization for both unknowns does not lead to big disadvantages, thus
\begin{equation*}
V_h=Q_h\equiv \text{span}\{\psi_i\}_{i=1}^{N_V}.
\end{equation*}

\subsection{Collocation of the BIE}
\label{sub:collocation_of_the_bie}

Two approaches are possible in order to discretize Eq.~\eqref{eq:characteristic-angle}. The \emph{Galerkin BEM} is a natural choice and it consists in writing the problem in variational form; however, this implies a second integration of weakly singular kernels, which would increase the computational complexity of the BEM. Therefore, we focus on \emph{collocation methods}. The idea is to directly use the discretized $\phi_h$ and $\gamma_h$ instead of the continuous $\phi$ and $\gamma$, and to collocate the BIE on a number of points equal to the number of unknowns, for example on the support of the Lagrangian basis functions.

We collocate Eq.~\eqref{eq:solid-angle}\footnote{It's simpler to impose the boundary conditions after assembling the linear system.} on the nodes $x_i$ of the boundary mesh, obtaining
\begin{align*}
\alpha(x_i)\phi(x_i)=&\sum_{K\in\Gamma_h}\int_K \gamma(y)\ G(x_i,y)\ \mathrm{d}S(y)\ + \\
& - \sum_{K\in\Gamma_h}\int_K \frac{\partial G}{\partial n}(x_i,y)\ \phi(y)\ \mathrm{d}S(y)
\end{align*}
for every $i=1,\dots,N_V$. Then by a quadrature formula which exploit the geometry of the boundary instead of using generic quadrature weights we get
\begin{align*}
\alpha(x_i)\phi(x_i)=&\sum_{K\in\Gamma_h}\sum_{q=1}^{N_q} \gamma(x_q) G(x_i,x_q) J^K(x_q)\ + \\
& - \sum_{K\in\Gamma_h}\sum_{q=1}^{N_q} \frac{\partial G}{\partial n}(x_i,x_q) \phi(x_q) J^K(x_q),
\end{align*}

where $J^K$ is the determinant of the first fundamental form for each reference cell $\hat{K}$. Now, as we anticipated before, we substitute $\phi$ with its approximation, i.e. $\phi(x_q)\approx \phi_h= \sum_j \phi_j \psi_j(x_q)$. We do the same thing for the flux $\frac{\partial\phi}{\partial n}$, and this yields
\begin{align*}
\alpha(x_i)\phi_j\psi_j(x_i)=&\sum_{K\in\Gamma_h}\sum_{q=1}^{N_q} \gamma_j \psi_j(x_q) G(x_i,x_q) J^K(x_q)\ + \\
& - \sum_{K\in\Gamma_h}\sum_{q=1}^{N_q} \frac{\partial G}{\partial n}(x_i,x_q) \phi_j \psi_j(x_q) J^K(x_q),
\end{align*}

and finally we obtain the linear system
\begin{equation}
\label{eq:first-LS}
(\alpha+N)\hat{\phi}-D\hat{\gamma}=0,
\end{equation}

where
\begin{itemize}
\item $\alpha$ is the diagonal matrix of values $\alpha(x_i)$;
\item $D_{ij}=\sum_K \sum_q G(x_i,x_q)\psi_j^q J^K$;
\item $N_{ij}=\sum_K \sum_q \frac{\partial G}{\partial n}(x_i,x_q)\psi_j^q J^K$;
\item $\hat{\phi}$, $\hat{\gamma}$ are the nodal values of the unknowns.
\end{itemize}

\subsection{Imposition of the Boundary Conditions}
\label{sub:imposition_of_the_boundary_conditions}

To impose the boundary condition for $\phi_h$ and $\gamma_h$, we define the additional vectors
\begin{subequations}
\label{eq:bc-LS}
\begin{align}
\label{eq:bc-LS1}
\tilde{\phi} &= 
\begin{cases}
0 & \text{on } \Gamma_D \\
\hat{\phi} & \text{on } \Gamma_N
\end{cases} \qquad &&
\tilde{\gamma} = 
\begin{cases}
\hat{\gamma} & \text{on } \Gamma_D \\
0 & \text{on }  \Gamma_N
\end{cases} \\
\label{eq:bc-LS2}
\bar{\phi} &= 
\begin{cases}
f_D & \text{on }  \Gamma_D \\
0 & \text{on }  \Gamma_N
\end{cases} \qquad &&
\bar{\gamma} = 
\begin{cases}
0 & \text{on }  \Gamma_D \\
f_N & \text{on } \Gamma_N
\end{cases}
\end{align}
\end{subequations}

With these defintions we can decouple the action of the two unknowns: while \eqref{eq:bc-LS1} clarify the real unknowns, \eqref{eq:bc-LS2} help setting the known values. The decoupled system is 
\begin{equation*}
\begin{cases}
(\alpha+N)\hat{\phi}-D\hat{\gamma}=0 & \text{on }\Gamma_D \\
(\alpha+N)\hat{\phi}-D\hat{\gamma}=0 & \text{on }\Gamma_N
\end{cases}
\end{equation*}
therefore
\begin{equation}
\label{eq:final-LS-with-underbracket}
\underbracket[0.5pt]{(\alpha+N)\tilde{\phi}-D\tilde{\gamma}}_{A\tilde{t}}=\underbracket[0.5pt]{-(\alpha+N)\bar{\phi}+D\bar{\gamma}}_{b}.
\end{equation}

To wrap up, we obtained the linear system $A\tilde{t}=b$, where $\tilde{t}$ simply groups the unknowns:
\begin{equation*}
\tilde{t}_i=
\begin{cases}
\hat{\gamma}_i &\text{if }x_i\in\Gamma_D \\
\hat{\phi}_i &\text{if }x_i\in\Gamma_N  
\end{cases}
\end{equation*}

Notice that the system in Eq.~\eqref{eq:final-LS-with-underbracket} is \emph{one to one} with Eq.~\eqref{eq:characteristic-angle}, as one would expect.

\subsection{Solution of the Linear System}
\label{sub:solution_of_the_linear_system}

In order to solve system \eqref{eq:final-LS-with-underbracket}, we don't assemble the matrix $A$ but we use the Generalized Minimal Residuals (GMRES), which is an iterative method based on Krylov subspace methods. By using least squares (and a preconditioner to accelerate the convergence), we minimize the residual $r^{(k)}=b-At^{(k)}$ till desired accuracy. This procedure works well in BEM frameworks, because it allows you to deal with nonsymmetric and fully populated matrices, and without explicitly forming them, saving a lot of memory.

\subsection{Post Process}
\label{sub:post_process}

We compute an error analysis to verify the proper convergence of the method and if necessary we proceed with a local refinement; then we recover the normal derivative $\frac{\partial \phi}{\partial n}$ (or simply the gradient $\nabla\phi$) by computing the derivative over Eq.~\eqref{eq:solid-angle}, and finally we output the solution.




